%% Generated by Sphinx.
\def\sphinxdocclass{article}
\documentclass[letterpaper,10pt,greek]{sphinxhowto}
\ifdefined\pdfpxdimen
   \let\sphinxpxdimen\pdfpxdimen\else\newdimen\sphinxpxdimen
\fi \sphinxpxdimen=.75bp\relax
\ifdefined\pdfimageresolution
    \pdfimageresolution= \numexpr \dimexpr1in\relax/\sphinxpxdimen\relax
\fi
%% let collapsible pdf bookmarks panel have high depth per default
\PassOptionsToPackage{bookmarksdepth=5}{hyperref}
%% turn off hyperref patch of \index as sphinx.xdy xindy module takes care of
%% suitable \hyperpage mark-up, working around hyperref-xindy incompatibility
\PassOptionsToPackage{hyperindex=false}{hyperref}
%% memoir class requires extra handling
\makeatletter\@ifclassloaded{memoir}
{\ifdefined\memhyperindexfalse\memhyperindexfalse\fi}{}\makeatother

\PassOptionsToPackage{warn}{textcomp}

\catcode`^^^^00a0\active\protected\def^^^^00a0{\leavevmode\nobreak\ }
\usepackage{cmap}
\usepackage{fontspec}
\defaultfontfeatures[\rmfamily,\sffamily,\ttfamily]{}
\usepackage{amsmath,amssymb,amstext}
\usepackage{polyglossia}
\setmainlanguage{greek}



\setmainfont{FreeSerif}[
  Extension      = .otf,
  UprightFont    = *,
  ItalicFont     = *Italic,
  BoldFont       = *Bold,
  BoldItalicFont = *BoldItalic
]
\setsansfont{FreeSans}[
  Extension      = .otf,
  UprightFont    = *,
  ItalicFont     = *Oblique,
  BoldFont       = *Bold,
  BoldItalicFont = *BoldOblique,
]
\setmonofont{FreeMono}[
  Extension      = .otf,
  UprightFont    = *,
  ItalicFont     = *Oblique,
  BoldFont       = *Bold,
  BoldItalicFont = *BoldOblique,
]

\newfontfamily\greekfont{FreeSerif}
\newfontfamily\greekfontsf{FreeSans}
\newfontfamily\greekfonttt{FreeMono}


\usepackage[Sonny]{fncychap}
\ChNameVar{\Large\normalfont\sffamily}
\ChTitleVar{\Large\normalfont\sffamily}
\usepackage[,numfigreset=2,mathnumfig]{sphinx}

\fvset{fontsize=\small}
\usepackage{geometry}


% Include hyperref last.
\usepackage{hyperref}
% Fix anchor placement for figures with captions.
\usepackage{hypcap}% it must be loaded after hyperref.
% Set up styles of URL: it should be placed after hyperref.
\urlstyle{same}


\usepackage{sphinxmessages}
\setcounter{tocdepth}{1}


\usepackage{bbm}
\usepackage{amsmath}
\usepackage{amsfonts}
    

\title{Τεκμηρίωση Υπολ. Συστημάτων και Υπηρεσιών}
\date{14 Ιουλίου 2022}
\release{}
\author{Δημήτρης Καλοψικάκης}
\newcommand{\sphinxlogo}{\vbox{}}
\renewcommand{\releasename}{}
\makeindex
\begin{document}

\pagestyle{empty}
\sphinxmaketitle
\pagestyle{plain}
\sphinxtableofcontents
\pagestyle{normal}
\phantomsection\label{\detokenize{index::doc}}


\sphinxstepscope


\section{Τρόπος Διεκπεραίωσης Εργασιών και Αιτημάτων}
\label{\detokenize{PolicyTicket:id1}}\label{\detokenize{PolicyTicket::doc}}
\sphinxAtStartPar
Στο κείμενο αυτό περιγράφεται ο τρόπος με τον οποίο
διεκπεραιώνονται οι εργασίες και τα αιτήματα που σχετίζονται
με τη Διαχείριση Συστημάτων Η/Υ στο Τμήμα Μαθηματικων και Εφαρμοσμένων Μαθηματικών.

\sphinxAtStartPar
Βασικές Αρχές και Υποθέσεις:
\begin{enumerate}
\sphinxsetlistlabels{\arabic}{enumi}{enumii}{}{.}%
\item {} 
\sphinxAtStartPar
Όλες οι εργασίες γίνονται απαρέγκλιτα εντός του ωραρίου εργασίας.

\item {} 
\sphinxAtStartPar
Εργασίες εκτός ωραρίου μπορούν να γίνονται μόνο στο πλαίσιο που ορίζεται από την εργασιακή νομοθεσία (υπερωρίες κλπ).

\item {} 
\sphinxAtStartPar
Υποθέτουμε ότι υπάρχει ισχυρή ανθρώπινη διάθεση για «γίνει η δουλειά στην ώρα της», δηλαδή να λειτουργήσει ο οργανισμός.

\item {} 
\sphinxAtStartPar
Η καταγραφή μιας εργασίας είναι \sphinxstylestrong{τουλάχιστον} το ίδιο σημαντική με την υλοποίηση της εργασίας. Μια μη καταγεγραμμένη εργασία είναι χαμένη εργασία. Η καταγραφή μιας εργασίας που δεν έγινε, μπορεί να βοηθήσει και να επιταχύνει πολλές άλλες παρόμοιες εργασίες.

\item {} 
\sphinxAtStartPar
Τα \sphinxstylestrong{επείγοντα} (βλ. παρακάτω) μπαίνουν σε άμεση προτεραιότητα, κατ” εξαίρεση οποιασδήποτε διαδικασίας. \sphinxstylestrong{Αυτό δεν τα εξαιρεί από την καταγραφή}· τουναντίον.

\item {} 
\sphinxAtStartPar
Τα \sphinxstylestrong{επείγοντα έχουν μηδενική πιθανότητα}, αλλιώς κάτι δεν πάει καλά στη λειτουργία του οργανισμού. Δηλαδή, η συχνότητα των επειγόντων πρέπει να είναι \sphinxstylestrong{σχεδόν ποτέ}, διαφορετικά κάνουμε κάτι \sphinxstyleemphasis{πολύ λάθος}.

\item {} 
\sphinxAtStartPar
Εκεί που χωλαίνει η διαδικασία επεμβαίνει ο ανθρώπινος παράγοντας με καλή και δημιουργική διάθεση· με στόχο, όμως, τη βελτίωση της διαδικασίας και όχι ένα προσωρινό μπάλωμα για να «ξεμπλέχνομενε».

\end{enumerate}

\sphinxAtStartPar
Κάποιοι ορισμοί:
\begin{enumerate}
\sphinxsetlistlabels{\arabic}{enumi}{enumii}{}{.}%
\item {} 
\sphinxAtStartPar
\sphinxstylestrong{Επείγον} είναι οτιδήποτε σχετίζεται με μια κατάσταση κατά την οποία η αξία το Οργανισμού μειώνεται ραγδαία. Παράδειγμα: πυρκαγιά, πλυμμήρα, κυβερνοεπίθεση, βλάβη κρίσιμου διακομιστή, κάποια άμεση απαίτηση από το βαθύ κράτος, κλπ

\item {} 
\sphinxAtStartPar
\sphinxstylestrong{Διοίκηση} του Τμήματος είναι οι γραμματείες και ο πρόεδρος.

\item {} 
\sphinxAtStartPar
Ο αδόκιμος ξενικός όρος \sphinxstylestrong{ticket}, χρησιμοποιείται στο εξής για «ηλεκτρονικά αιτήματα», «εργασίες», κλπ.

\item {} 
\sphinxAtStartPar
Αντίστοιχα, ο όρος \sphinxstylestrong{requester}, χρησιμοποιείται για τον «αιτούμενο» ή για αυτόν που εισάγει μια εργασία προς διεκπεραίωση.

\item {} 
\sphinxAtStartPar
Με τον όρο \sphinxstylestrong{followup}, εννοούμε τόσο την καταγραφή ενός ενδιάμεσου σταδίου διεκπεραίωσης ενός ticket, όσο και ένα ενημερωτικό μήνυμα από/προς τον requester.

\end{enumerate}


\subsection{Προτεραιότητες}
\label{\detokenize{PolicyTicket:id2}}
\sphinxAtStartPar
Η προτεραιοποίηση εργασιών, εν γένει είναι ζήτημα περίπλοκο και πολυδιάστατο
και οι συνιστώσες του είναι συχνά αντικρουόμενες.


\subsubsection{Διάσταση Α: Οι βασικές προτεραιότητες}
\label{\detokenize{PolicyTicket:id3}}\begin{enumerate}
\sphinxsetlistlabels{\arabic}{enumi}{enumii}{}{.}%
\setcounter{enumi}{-1}
\item {} 
\sphinxAtStartPar
Επείγοντα.

\item {} 
\sphinxAtStartPar
Αιτήματα Διοίκησης.

\item {} 
\sphinxAtStartPar
Εργασίες Διαχείρισης Συστημάτων.

\item {} 
\sphinxAtStartPar
Εργασίες που αφορούν μαθήματα.

\item {} 
\sphinxAtStartPar
Αιτήματα επισκεπτών.

\item {} 
\sphinxAtStartPar
Αιτήματα ακαδημαϊκού και διοικητικού/τεχνικού Προσωπικού.

\item {} 
\sphinxAtStartPar
Αιτήματα φοιτητών.

\end{enumerate}


\subsubsection{Διάσταση Β: \protect\(+\infty > 1\protect\)}
\label{\detokenize{PolicyTicket:infty-1}}
\sphinxAtStartPar
Ό,τι αφορά σε πολλούς είναι σημαντικότερο από ό,τι αφορά σε έναν.


\subsubsection{Διάσταση Γ: Όχι στην πείνα των μικρών}
\label{\detokenize{PolicyTicket:id4}}
\sphinxAtStartPar
Μικρά αιτήματα πρέπει να εξυπηρετούνται σε ανάλογα μικρό χρόνο.
Αλλιώς, είναι άδικο και –σωστά– αυξάνει πολύ η δυσαρέσκεια
και τότε πολλά μικρά και ασήμαντα μετατρέπονται σε επείγοντα
(δηλ. ραγδαία μείωση αξίας οργανισμού).

\sphinxAtStartPar
Π.χ.: μια προσθήκη σε μια mailing list δεν πρέπει να περιμένει
το ticket της «μελέτης και ανάπτυξης database server»
που θα πάρει μήνες.


\subsubsection{Διάσταση Δ: FIFO}
\label{\detokenize{PolicyTicket:fifo}}
\sphinxAtStartPar
Τα αιτήματα πρέπει να διεκπεραιώνονται με τη σειρά τους.
Δηλαδή, ανάμεσα σε αιτήματα ίσης προτεραιότητας , τα παλιότερα προηγούνται των νεότερων.


\subsubsection{Διάσταση Ε: Χρόνος}
\label{\detokenize{PolicyTicket:id5}}\begin{enumerate}
\sphinxsetlistlabels{\arabic}{enumi}{enumii}{}{.}%
\item {} 
\sphinxAtStartPar
Η προτεραιότητα αιτημάτων με σκληρές προθεσμίες, αυξάνει σημαντικά καθώς πλησιάζει το deadline.

\item {} 
\sphinxAtStartPar
Η προτεραιότητα κάθε αιτήματος αυξάνει με την απόσταση από τον χρόνο υποβολής του αιτήματος.

\end{enumerate}


\subsection{Μέθοδος Εργασίας}
\label{\detokenize{PolicyTicket:id6}}
\sphinxAtStartPar
Τα δύο βασικά στοιχεία περιεχομένου ενός ticket, πέρα από τον τίτλο του,
είναι η περιγραφή του και τα followup. Η περιγραφή καθορίζεται από τον
requester κατά τη δημιουργία του ticket. Ένα followup είναι είτε ένα
ενημερωτικό μήνυμα από ή προς τον requester ή μια καταγραφή με χρήσιμη
πληροφορία για κάποιον μελλοντικό αναγνώστη του ticket –π.χ. έναν
τεχνικό που προσπαθεί να λύσει ένα παρόμοιο πρόβλημα.

\sphinxAtStartPar
Τα followup επικοινωνούνται με email τόσο στον requester όσο και
στον τεχνικό στον οποίο έχει ανατεθεί το ticket. Επομένως, δεν
πρέπει να είναι ούτε πολλά, ούτε λίγα, αλλά ακριβώς όσα χρειάζονται
για να μην υπάρχει πληροφοριακός θόρυβος.

\sphinxAtStartPar
Χρησιμοποιούμε τρεις καταστάσεις ενός ticket:
\begin{itemize}
\item {} 
\sphinxAtStartPar
\sphinxstylestrong{New}: Μόλις δημουργηθεί ένα ticket τίθεται στην κατάσταση New.

\item {} 
\sphinxAtStartPar
\sphinxstylestrong{Processing}: Μόλις γίνει ανάθεση σε έναν τεχνικό, τίθεται στην κατάσταση Processing.

\item {} 
\sphinxAtStartPar
\sphinxstylestrong{Closed}: Μόλις ολοκληρωθεί η διεκπεραίωσή του, τίθεται σε κατάσταση Closed.

\end{itemize}

\sphinxAtStartPar
Η βασική μέθοδος διεκπεραίωσης ενός ticket είναι η εξής απλή:
\begin{enumerate}
\sphinxsetlistlabels{\Alph}{enumi}{enumii}{}{.}%
\item {} 
\sphinxAtStartPar
Δημιουργία ενός ticket από κάποιον requester.

\item {} 
\sphinxAtStartPar
Το ticket ανατίθεται σε τεχνικό και μπαίνει σε κατάσταση Processing.

\item {} 
\sphinxAtStartPar
Ο Τεχνικός ξεκινά τη διαδικασία επεξεργασίας:
\begin{enumerate}
\sphinxsetlistlabels{\arabic}{enumii}{enumiii}{}{.}%
\item {} 
\sphinxAtStartPar
αν χρειάζεται, επικοινωνεί με τον requester για τυχόν διευκρινίσεις ή επιπρόσθετες πληροφορίες.

\item {} 
\sphinxAtStartPar
καταγράφει την επικοινωνία σε followup μαζί την όποια χρήσιμη πληροφορία.

\item {} 
\sphinxAtStartPar
εκπονεί μέρος της εργασίας

\item {} 
\sphinxAtStartPar
προσθέτει ένα followup ώστε αφενός να ενημερωθεί ο requester για την πρόοδο του αιτήματός του και αφετέρου να καταγραφεί και ο τρόπος που έγινε η εργασία, τα προβλήματα που αντιμετωπίστηκαν και ο τρόπος λύθηκαν ή παρακάμφθηκαν.

\item {} 
\sphinxAtStartPar
αν το ticket δεν έχει ολοκληρωθεί (ο τεχνικός), ίσως σε δεύτερο χρόνο, επανέρχεται στο πρώτο βήμα της επεξεργασίας.

\end{enumerate}

\item {} 
\sphinxAtStartPar
Το ticket τιθεται σε κατάσταση Closed.

\end{enumerate}

\sphinxincludegraphics[]{graphviz-304b7ca420d7fda58873a468f53c7e9e4f2c3647.pdf}

\sphinxstepscope


\section{Ηλεκτρονικά Αιτήματα}
\label{\detokenize{HowToTicket:id1}}\label{\detokenize{HowToTicket::doc}}
\sphinxAtStartPar
Οποιοδήποτε αίτημα προς τη Διαχείριση Υπολογιστικών Συστημάτων
του Τμήματος, γίνεται ηλεκτρονικά μέσω της ιστοσελίδας
\sphinxurl{https://techsupport.math.uoc.gr}. Η πρόσβαση στη σελίδα αυτή
γίνεται μέσω ιδρυματικού λογαριασμού (username: \sphinxcode{\sphinxupquote{*@*uoc.gr}}),
όπου επίσης γίνεται και η παρακολούθηση της εξυπηρέτησης του
κάθε αιτήματος.

\sphinxAtStartPar
Εναλλακτικά, αιτήματα μπορούν να υποβληθούν με απλή αποστολή
μηνύματος στη διεύθυνση: \sphinxhref{mailto:techsupport@math.uoc.gr}{techsupport@math.uoc.gr}.
Η διεύθυνση αποστολέα πρέπει να είναι είτειδρυματική από το
Πανεπιστήμιο Κρήτης, είτε εξωτερική από το gmail.

\sphinxstepscope


\section{Πώς θα εγκαταστήσω LaTeX}
\label{\detokenize{HowToInstallLaTeX:latex}}\label{\detokenize{HowToInstallLaTeX::doc}}
\sphinxAtStartPar
Για να μπορέσουμε να γράψουμε κείμενα σε LaTeΧ, τα εργαλεία που
χρειαζόμαστε είναι δύο: (1) έναν κειμενογράφο
στον οποίο γράφουμε το κείμενο LaTeX (2) το λογισμικό που μεταφράζει τις εντολές
LaTeX σε αρχεία div, postscript, pdf, κλπ.

\sphinxAtStartPar
Τα εργαλεία που επιλέγοντα σε αυτόν τον οδηγό είναι:
\begin{itemize}
\item {} 
\sphinxAtStartPar
Επεξεργαστής κειμένου: \sphinxhref{https://www.xm1math.net/texmaker/}{Texmaker}

\item {} 
\sphinxAtStartPar
Το λογισμικό LaTex: \sphinxhref{https://www.tug.org/texlive/}{TeXLive}

\end{itemize}

\sphinxAtStartPar
Το TeXLive διανέμεται ως ένα εικονικό DVD (iso image). Για να μπορέσουμε
να το χρησιμοποιήσουμε έχουμε δύο επιλογές· είτε να το εγγράψουμε (κάψουμε)
σε ένα DVD είτε να το απεικονίσουμε (mount) σε έναν φάκελο (linux) ή σε
ένα εικονικό drive (windows).


\subsection{Σε Linux Debian/Ubuntu}
\label{\detokenize{HowToInstallLaTeX:linux-debian-ubuntu}}\begin{itemize}
\item {} 
\sphinxAtStartPar
\sphinxcode{\sphinxupquote{apt\sphinxhyphen{}get install texlive\sphinxhyphen{}full}}

\item {} 
\sphinxAtStartPar
\sphinxcode{\sphinxupquote{apt\sphinxhyphen{}get install texmaker}}

\end{itemize}


\subsection{Σε οποιοδήποτε Linux}
\label{\detokenize{HowToInstallLaTeX:linux}}\begin{enumerate}
\sphinxsetlistlabels{\arabic}{enumi}{enumii}{}{.}%
\item {} 
\sphinxAtStartPar
Download texlive iso image

\item {} 
\sphinxAtStartPar
mount iso image

\item {} 
\sphinxAtStartPar
Install texlive

\item {} 
\sphinxAtStartPar
Download and install Texmaker

\end{enumerate}


\subsection{Σε Windows}
\label{\detokenize{HowToInstallLaTeX:windows}}
\sphinxAtStartPar
Στα λειτουργικά συστήματα MS Windows, για να απεικονιστεί το iso image
σε κάποιο εικονικό drive θα πρέπει να εγκαταστήσουμε κάποιο πρόγραμμα
το οποίο κάνει αυτή τη δουλειά.

\sphinxAtStartPar
Συνολικά, τα βήματα που ακολουθούμε είναι τα εξής:
\begin{enumerate}
\sphinxsetlistlabels{\arabic}{enumi}{enumii}{}{.}%
\item {} 
\sphinxAtStartPar
Κατεβάζουμε και εγκαιστούμε το πρόγραμμα \sphinxhref{https://wincdemu.sysprogs.org/}{WinCDEmu}.

\item {} 
\sphinxAtStartPar
Κατεβάζουμε το iso image για το \sphinxhref{http://ftp.ntua.gr/pub/tex/systems/texlive/Images/}{TeXLive} (θα πάρει από 30 λεπτά μέχρι 2 ώρες, αναλόγως την ταχύτητα του δικτύου)

\item {} 
\sphinxAtStartPar
Το αρχείο που κατεβάσαμε το ανοίγουμε με το winCDEmu: δεξί κλικ και «Open with–> WinCDEmu»

\item {} 
\sphinxAtStartPar
Ανοίγουμε το εικονικό drive που έχει δημιουργηθεί και εκτελούμε το πρόγραμμα \sphinxcode{\sphinxupquote{install\sphinxhyphen{}tl\sphinxhyphen{}windows}} (θα πάρει μία με μιάμιση ώρα, αναλόγως την ταχύτητα του υπολογιστή)

\item {} 
\sphinxAtStartPar
Όταν τελειώσει, κατεβάζουμε το \sphinxhref{https://www.xm1math.net/texmaker/}{Texmaker} και το εγκαθιστούμε. Είναι σημαντικό ο Texmaker να εγκατασταθεί τελευταίος έτσι ώστε να ενημερωθεί για τα PATHs του TeXLive.

\end{enumerate}

\sphinxstepscope


\section{Πώς μεταφέρουμε μεγάλα αρχεία}
\label{\detokenize{HowToTransfer:id1}}\label{\detokenize{HowToTransfer::doc}}

\subsection{1ο Τρόπος: UoCTransfer (ΚΥΥΤΠΕ)}
\label{\detokenize{HowToTransfer:uoctransfer}}
\sphinxAtStartPar
Για τη μεταφορά μεγάλων αρχείων λειτουργεί ιδρυατική
υπηρεσία στη διεύθυνση:
\begin{itemize}
\item {} 
\sphinxAtStartPar
\sphinxurl{https://uoctransfer.ict.uoc.gr}

\end{itemize}


\subsection{2ος Τρόπος: OneDrive (Microsoft)}
\label{\detokenize{HowToTransfer:onedrive-microsoft}}
\sphinxAtStartPar
Τα μέλη του Πανεπιστημίου Κρήτης, έχουν πρόσβαση στο OneDrive της Microsoft
μέσω των ιδρυματικών τους λογαριασμών, στη διεύθυνση:
\begin{itemize}
\item {} 
\sphinxAtStartPar
\sphinxurl{https://delos365.grnet.gr}

\end{itemize}

\sphinxAtStartPar
Το OneDrive παρέχει τη δυνατότητα διαμοιρασμού (sharing) αρχείων ή φακέλων.



\renewcommand{\indexname}{Ευρετήριο}
\printindex
\end{document}